The computer architecture can be thought of as a hierarchy, as seen in Figure 1 \citep{Tanenbaum}.  Programmers and other humans interact with the higher levels of the hierarchy, while the machine itself interacts with the lower levels.  Digital logic is the foundation of this hierarchy.  At this level, logic and arithmetic are performed on binary values (zeros and ones).  Conceptually, this logic and arithmetic can be explained by gates and truth tables.  Gates are objects that take input values and produce an output value.  Truth tables list the possible combinations of inputs and the outputs of those combinations.  The basic gate, the NAND gate (Figure 2), creates the foundation of the entire computer architecture; every other gate can be built as combinations of NANDs, and any functionality of the computer architecture at higher levels can be broken down into functionality of NAND gates at the digital logic level.