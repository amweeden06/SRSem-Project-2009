% Aaron's poster for the Earlham Annual Research Conference
%
% Poster based on PetaKit "how" poster, which quotes:
% "All of the poster stuff was derived from:
%
% Template file for an a0 landscape poster.
% Written by Graeme, 2001-03 based on Norman's original microlensing
% poster.
%
% See discussion and documentation at
% <http://www.astro.gla.ac.uk/users/norman/docs/posters/> 
%
% $Id: poster.tex,v 1.6 2010/04/02 14:07:13 leemasa Exp $
%
% and work that Josh Hursey and Josh McCoy did on the b-and-t-gromacs and 
% Big-FE posters.  I have stripped the parts we don't use out.  

% You might find the 'draft' option to a0 poster useful if you have
% lots of graphics, because they can take some time to process and
% display. (\documentclass[a0,draft]{a0poster})"
%
\documentclass[a0]{a0poster}

\pagestyle{empty}
\setcounter{secnumdepth}{0}
\newcommand{\standardsize}{\fontsize{28}{33}\selectfont}

% The textpos package is necessary to position textblocks at arbitary 
% places on the page.
%
\usepackage[absolute]{textpos}

\usepackage{url}

% Graphics to include graphics. Times is nice on posters, but you
% might want to switch it off and go for CMR fonts.
%
\usepackage{graphicx,wrapfig,times}

% Avoid unnecessary hyphenation in paragraphs. Also requires
% \raggedright below, after \begin{document}.
%
\usepackage[none]{hyphenat}

% These colours are tried and tested for titles and headers. Don't
% over use color!
%
\usepackage{color}
\definecolor{DarkBlue}{rgb}{0.1,0.1,0.5}
\definecolor{Red}{rgb}{0.9,0.0,0.1}
\definecolor{Green}{rgb}{0.0,0.6,0.1}

% See documentation for a0poster class for the size options here
%
\let
\Textsize
\normalsize
\def\Head#1{\noindent\hbox to \hsize{\hfil{\LARGE\color{DarkBlue} #1}}\bigskip}
\def\LHead#1{\noindent{\LARGE\color{DarkBlue} #1}\bigskip}
\def\Subhead#1{\noindent{\large\color{DarkBlue} #1}}
\def\Title#1{\noindent{\VeryHuge\color{Green} #1}}

% Set up the grid
%
% Note that [40mm,40mm] is the margin round the edge of the page --
% it is _not_ the grid size. That is always defined as 
% PAGE_WIDTH/HGRID and PAGE_HEIGHT/VGRID. In this case we use
% 23 x 12. This gives us three columns of width 7 boxes, with a gap of
% width 1 in between them. 12 vertical boxes is a good number to work
% with.
%
% Note however that texblocks can be positioned fractionally as well,
% so really any convenient grid size can be used.
%
\TPGrid[40mm,40mm]{23}{12}      % 3 cols of width 7, plus 2 gaps width 1

\parindent=0pt
\parskip=0.5
\baselineskip

\begin{document}
\large
\raggedright

% Understanding textblocks is the key to being able to do a poster in
% LaTeX. In
%
%    \begin{textblock}{wid}(x,y)
%    ...
%    \end{textblock}
%
% the first argument gives the block width in units of the grid
% cells specified above in \TPGrid; the second gives the (x,y)
% position on the grid, with the y axis pointing down.
%
% You will have to do a lot of previewing to get everything in the 
% right place.
%
% Watch out for hyphenation in titles - LaTeX will do it
% but it looks awful.
%

% Title
%
\begin{textblock}{23}(0,0)
\Title{\italics[Computer City: Sewers]: An Educational Game to Teach Digital Logic}
\end{textblock}

% Authors and institutions
%
\begin{textblock}{20}(0,0.6)
\LHead{Aaron Weeden}
\hfil
\break
\textsl{Earlham College -- amweeden06@cs.earlham.edu}
\end{textblock}

% Abstract
% 
\begin{textblock}{23}(0,1.2)
\large
\input{abstract.tex}
\normalsize
\end{textblock}

% Logos at top of each column.
% 
%\begin{textblock}{2}(2.2,3.2)
%\includegraphics{../artwork/littlefe.eps}
%\end{textblock}
%
%\begin{textblock}{2}(10.9,3.1)
%\includegraphics{../artwork/bccd.eps}
%\end{textblock}
%
%\begin{textblock}{2}(18.0,3.4)
%\includegraphics{../artwork/cserd.eps}
%\end{textblock}

% First column 
% 
\begin{textblock}{6}(0,2.6)
\hrule
\medskip
\LHead{Technical Overview}
\standardsize
\input{technical.tex}
\end{textblock}

% Second column 
% 
\begin{textblock}{11}(6,2.7)
\begin{center}
\includegraphics[scale=2.4]{../artwork/PetaKit.eps}
~\\
\vspace{10mm}
~\\
\includegraphics[scale=3]{../artwork/strong.ps}

A performance statistics graph produced with PetaKit
\end{center}
\end{textblock}

% Third column 
% 
\begin{textblock}{5.5}(17,2.6)
\hrule
\standardsize
\medskip
\LHead{Design}

\input{design.tex}  
\end{textblock}

% Logos along bottom edge of the poster.
%
\def\PosterBottom{11.5}
\def\TopRow{11}
\def\LogoHeight{40mm}
\def\LogoSmall{25mm}

\begin{textblock}{2}(0,\PosterBottom)
\includegraphics[height=\LogoHeight]{../artwork/teragrid.eps}
\end{textblock}

\begin{textblock}{2}(2.7,\PosterBottom)
\includegraphics[height=\LogoHeight]{../artwork/kean.eps}
\end{textblock}

\begin{textblock}{2}(4,\PosterBottom)
\includegraphics[height=\LogoHeight]{../artwork/Earlham.eps}
\end{textblock}

\begin{textblock}{2}(7.7,11.4)
\includegraphics[height=60mm]{../artwork/SC09.eps}
\end{textblock}

\begin{textblock}{2}(9,\PosterBottom)
\includegraphics[height=\LogoHeight]{../artwork/shodor.eps}
\end{textblock}

\begin{textblock}{2}(12.6,\PosterBottom)
\includegraphics[height=\LogoHeight]{../artwork/Blue_Waters.eps}
\end{textblock}

\begin{textblock}{2}(15.6,\PosterBottom)
\includegraphics[height=50mm]{../artwork/oscer.eps}
\end{textblock}

\begin{textblock}{2}(16.8,\PosterBottom)
\includegraphics[height=\LogoHeight]{../artwork/IU.eps}
\end{textblock}

\begin{textblock}{2}(17.7,\PosterBottom)
\includegraphics[height=\LogoHeight]{../artwork/PSC.eps}
\end{textblock}

\begin{textblock}{2}(21.5,\PosterBottom)
\includegraphics[height=\LogoHeight]{../artwork/Intel.eps}
\end{textblock}

\end{document}
